\documentclass[11pt]{article}
%prepared in AMSLaTeX, under LaTeX2e
\addtolength{\oddsidemargin}{-.75in} 
\addtolength{\evensidemargin}{-.75in}
\addtolength{\topmargin}{-.6in}
\addtolength{\textwidth}{1.4in}
\addtolength{\textheight}{1.3in}

\renewcommand{\baselinestretch}{1.06}

\usepackage{wrapfig,fancyvrb,xspace}
\usepackage{palatino,amsmath,amssymb,amsthm,bm}
\usepackage[final]{graphicx}
\usepackage[pdftex, colorlinks=true, plainpages=false, linkcolor=blue, citecolor=red, urlcolor=blue]{hyperref}

% macros
\newcommand{\bn}{\mathbf{n}}
\newcommand{\bq}{\mathbf{q}}
\newcommand{\bu}{\mathbf{u}}
\newcommand{\bw}{\mathbf{w}}

\newcommand{\bX}{\mathbf{X}}

\newcommand{\cH}{\mathcal{H}}
\newcommand{\cK}{\mathcal{K}}
\newcommand{\cV}{\mathcal{V}}

\newcommand{\RR}{\mathbb{R}}

\newcommand{\Div}{\nabla\cdot}
\newcommand{\eps}{\epsilon}
\newcommand{\grad}{\nabla}
\newcommand{\lam}{\lambda}

\newcommand{\ds}{\displaystyle}


\title{A porous media test case in Firedrake}
\author{Ed Bueler}
\date{\today}

\begin{document}
\maketitle
%\begin{abstract}
%FIXME
%\end{abstract}

\thispagestyle{empty}

\section{Draft model}

Suppose $\Omega$ is a 2-dimensional domain such as a square.  We will use $x$ for the horizontal and $z$ for the vertical coordinate.  We assume $z$ is measured positive upward.  Within $\Omega$ we assume there is a matrix of porous material with variable properties of porosity $\phi(x,z)$ and permeability $k(x,z)$; these are assumed independent of time.

A gas flows through that medium.  We will take various properties of the gas to be positive constants: $\mu$ is the dynamic viscosity, $R$ is the gas constant, $T$ is the absolute temperature, and $M$ is the molar mass.  The following system of equations, a mathematical model, applies to determing the evolution of the density $\rho(t,x,z)$, pressure $P(t,x,z)$, and vector volumetric flux $\bq(t,x,z)$:\footnote{TS: Please check that this is the model you want?}
\begin{subequations}
\label{eq:pmtime:early}
\begin{align}
\frac{\partial}{\partial t} \left(\rho \phi\right) + \Div \left(\rho\, \bq\right) &= 0 \label{eq:masscont} \\
\bq &= - \frac{k}{\mu} \grad\left(P + \rho g z\right) \label{eq:darcy} \\
P &= \frac{RT}{M} \rho \label{eq:idealgas}
\end{align}
\end{subequations}

Regarding the above system we can make comments.  We are assuming no mass sources or sinks in \eqref{eq:masscont}, otherwise there could be a function on the right.  From the static pressure $P_0=-\rho g z$ note that $P+\rho g z = P-P_0$, we see that the flow in \eqref{eq:darcy} is driven by deviations from static pressure.  The volumetric flux $\bq$ is related to the vector velocity $\bu$ by $\bq = \phi \bu$.  Velocity may be helpful in diagnosing the solution but it is not needed in stating the above equations.

It is straightforward to eliminate $\bq$ and $P$, using also the time-independence of $\phi$:
\begin{equation}
\phi \frac{\partial \rho}{\partial t} - \frac{1}{\mu} \Div \left(k \rho \grad\left(\frac{RT}{M} \rho + \rho g z\right)\right) = 0 \label{eq:pmtime}
\end{equation}


\section{Steady state test case}

The porous medium system \eqref{eq:pmtime:early} or \eqref{eq:pmtime} is time dependent.  It has an obvious steady-state form:
\begin{equation}
- \Div \left(k \rho \grad\left(\frac{RT}{M} \rho + \rho g z\right)\right) = 0 \label{eq:pm:withunits}
\end{equation}
Note that $\phi,\mu$ are not parameters in this form.  Let us choose units so that the remaining constants are all one: $R=T=M=q=1$, and furthermore for this test case we assume $k=1$ is constant.  Then the equation is simply
\begin{equation}
- \Div \left(\rho \grad\left(\rho + \rho z\right)\right) = 0 \label{eq:pm:strong}
\end{equation}

Let us assume $\Omega$ is the unit square $0<x<1,0<z<1$.  Matching the Firedrake convention from \texttt{UnitSquareMesh()}, the sides are indexed (1,2,3,4) for (left,right,bottom,top) respectively.

We need boundary conditions.\footnote{TS: I have not idea what you want, actually.}  On the top let us suppose there is known density of one.  On the sides and the base let us assume there is no flux.  Thus
\begin{subequations}
\begin{align}
\rho &= 1 & &(4=\text{top}) \label{eq:bc:dirichlet} \\
\bq \cdot \bn = - \grad(\rho + \rho z) \cdot \bn &= 0 & &(1,2,3=\text{left},\text{right},\text{bottom}) \label{eq:bc:neumann}
\end{align}
\end{subequations}

In the above $\rho(x,z)$ is the scalar unknown.  Now we can derive the weak form by multiplying by a test function $w(x,z)$ and integrating over $\Omega$:
    $$\int_\Omega- \Div \left(\rho \grad\left(\rho + \rho z\right)\right) w = 0.$$
Integration by parts, i.e.~the product rule plus the divergence theorem, gives:
	$$- \int_{\partial\Omega} \rho w \grad\left(\rho + \rho z\right) \cdot \bn + \int_\Omega \rho \grad\left(\rho + \rho z\right) \cdot \grad w = 0.$$
By \eqref{eq:bc:neumann} the 1,2,3 parts of the boundary $\partial\Omega$ give integral zero, thus
	$$- \int_{\partial_4\Omega} \rho w \grad\left(\rho + \rho z\right) \cdot \bn + \int_\Omega \rho \grad\left(\rho + \rho z\right) \cdot \grad w = 0.$$

To enforce the boundary condition on the top boundary we must assume that the test functions $w$ satisfy zero along that.  Thus the entire boundary integral is zero, and the weak form for this test case is just
\begin{equation}
\int_\Omega \rho \grad\left(\rho + \rho z\right) \cdot \grad w = 0.\label{eq:pm:weak}
\end{equation}

\end{document}

