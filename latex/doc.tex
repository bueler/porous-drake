\documentclass[11pt]{article}
%prepared in AMSLaTeX, under LaTeX2e
\addtolength{\oddsidemargin}{-.75in} 
\addtolength{\evensidemargin}{-.75in}
\addtolength{\topmargin}{-.6in}
\addtolength{\textwidth}{1.4in}
\addtolength{\textheight}{1.3in}

\renewcommand{\baselinestretch}{1.06}

\usepackage{wrapfig,fancyvrb,xspace}
\usepackage{palatino,amsmath,amssymb,amsthm,bm}
\usepackage[final]{graphicx}
\usepackage[pdftex, colorlinks=true, plainpages=false, linkcolor=blue, citecolor=red, urlcolor=blue]{hyperref}

% macros
\newcommand{\bn}{\mathbf{n}}
\newcommand{\bq}{\mathbf{q}}
\newcommand{\bu}{\mathbf{u}}
\newcommand{\bw}{\mathbf{w}}
\newcommand{\bx}{\mathbf{x}}

\newcommand{\bX}{\mathbf{X}}

\newcommand{\cH}{\mathcal{H}}
\newcommand{\cK}{\mathcal{K}}
\newcommand{\cV}{\mathcal{V}}

\newcommand{\RR}{\mathbb{R}}

\newcommand{\Div}{\nabla\cdot}
\newcommand{\eps}{\epsilon}
\newcommand{\grad}{\nabla}
\newcommand{\lam}{\lambda}

\newcommand{\ds}{\displaystyle}


\title{A porous media test case in Firedrake}
\author{Ed Bueler}
\date{\today}

\begin{document}
\maketitle
%\begin{abstract}
%FIXME
%\end{abstract}

\thispagestyle{empty}

\section{A (draft) porous medium model for Darcy-type gas flow}

Suppose $\Omega$ is a 2-dimensional domain such as a square.  We will use $x$ for the horizontal and $z$ for the vertical coordinate.  We assume $z$ is measured positive upward.  Within $\Omega$ we assume there is a matrix of porous material with variable properties of porosity $\phi(x,z)$ and permeability $k(x,z)$; these are assumed independent of time.

A gas flows through that medium.  We will take various properties of the gas to be positive constants: $\mu$ is the dynamic viscosity, $R$ is the gas constant, $T$ is the absolute temperature, and $M$ is the molar mass.  The following system of equations, a mathematical model, applies to determing the evolution of the density $\rho(t,x,z)$, pressure $P(t,x,z)$, and vector volumetric flux $\bq(t,x,z)$:\footnote{TS: Please check that this is the model you want?}
\begin{subequations}
\label{eq:pmtime:early}
\begin{align}
\frac{\partial}{\partial t} \left(\rho \phi\right) + \Div \left(\rho\, \bq\right) &= 0 \label{eq:masscont} \\
\bq &= - \frac{k}{\mu} \grad\left(P + \rho g z\right) \label{eq:darcy} \\
P &= \frac{RT}{M} \rho \label{eq:idealgas}
\end{align}
\end{subequations}

Regarding the above system we can make comments.  To be physical the density must be non-negative: $\rho\ge 0$.  We are assuming no mass sources or sinks in \eqref{eq:masscont}, otherwise there could be a function on the right.  From the static pressure $P_0=-\rho g z$ note that $P+\rho g z = P-P_0$, we see that the flow in \eqref{eq:darcy} is driven by deviations from static pressure.  The volumetric flux $\bq$ is related to the vector velocity $\bu$ by $\bq = \phi \bu$.  (Velocity may be helpful in diagnosing the solution but it is not needed to state the above.)

It is straightforward to eliminate $\bq$ and $P$, using also the time-independence of $\phi$.  We also simplify the name of the nontrivial ideal gas law constant:
\begin{equation}
c = \frac{RT}{M}.  \label{eq:crename}
\end{equation}
Thus:
\begin{equation}
\phi \frac{\partial \rho}{\partial t} - \frac{1}{\mu} \Div \left(k \rho \grad\left(c \rho + \rho g z\right)\right) = 0.  \qquad \text{\emph{(porous medium equation)}} \label{eq:pmtime}
\end{equation}
This form is comparable to the better-known heat equation
\begin{equation}
\frac{\partial u}{\partial t} - \Div(\grad u) = 0. \qquad \text{\emph{(heat equation)}}\label{eq:heattime}
\end{equation}

An important property of heat equation \eqref{eq:heattime} is that at each time the operator $\Div \grad = \grad^2$ acts as a kind of invertible matrix, which assists in finding the solution $u$.  Since $\rho$ appears as a flux coefficient in the porous medium equation, i.e.~the copy of $\rho$ just inside the divergence in \eqref{eq:pmtime}, this invertibility will be lost if $\rho\to 0$ somewhere.  This is called ``degeneration'' of the diffusivity.  It is a well-known property of the porous medium equation which must be taken seriously when solving numerically.


\section{Steady state test cases}

The porous medium equation \eqref{eq:pmtime} is time dependent.  It has an obvious steady-state form found by setting $\partial \rho/\partial t = 0$:
\begin{equation}
- \Div \left(k \rho \grad\left(c \rho + \rho g z\right)\right) = 0 \label{eq:pm:strong}
\end{equation}
In equation \eqref{eq:pm:strong}, the strong form, the scalar function $\rho(x,z)$ is unknown.  Note that $\phi,\mu$ are not parameters in this time-independent and source-free situation.  

To derive the weak form we first multiply by a test function $w(x,z)$ and integrate over $\Omega$:
    $$\int_\Omega- \Div \left(k \rho \grad\left(c \rho + \rho g z\right)\right) w = 0.$$
Integration by parts, i.e.~the product rule plus the divergence theorem, gives:
\begin{equation}
- \int_{\partial\Omega} k \rho w \grad\left(c \rho + \rho g z\right) \cdot \bn + \int_\Omega k \rho \grad\left(c \rho + \rho g z\right) \cdot \grad w = 0.\label{eq:pm:preliminaryweak}
\end{equation}
Equation \eqref{eq:pm:preliminaryweak} is a preliminary weak form, into which we must put some boundary conditions.\footnote{TS: I have no idea what boundary conditions you want, actually.}

For now here are two proposed, concrete test cases.  Here $\Omega$ is the unit square $0<x<1,0<z<1$.  Matching the Firedrake convention from \texttt{UnitSquareMesh()}, the sides are indexed (1,2,3,4) for (left,right,bottom,top) respectively.

\begin{enumerate}
\item On the top side of the square $\Omega$, suppose the density has a known value of one.  On the sides and the base let us assume there is no flux.  Thus
\begin{subequations}
\begin{align}
\rho &= 1 & &(4=\text{top}) \label{eq:onebc:dirichlet} \\
\bq \cdot \bn = - \frac{k}{\mu} \grad(c\rho + \rho g z) \cdot \bn &= 0 & &(1,2,3=\text{left},\text{right},\text{bottom}) \label{eq:onebc:noflux}
\end{align}
\end{subequations}
By \eqref{eq:onebc:noflux} the 1,2,3 parts of the boundary integral in \eqref{eq:pm:preliminaryweak} are zero, thus
	$$- \int_{\partial_4\Omega} k \rho w \grad\left(c \rho + \rho g z\right) \cdot \bn + \int_\Omega k \rho \grad\left(c \rho + \rho g z\right) \cdot \grad w = 0.$$

To enforce the condition on the top boundary $\partial_4\Omega$ we must assume that the test functions $w$ satisfy zero along there.  Denote $H_D^1(\Omega)$ for this space of test functions which are zero along the Dirichlet boundary.  The entire boundary integral is now zero, and the weak form is just
\begin{equation}
\int_\Omega k\rho \grad\left(c\rho + \rho g z\right) \cdot \grad w = 0 \qquad \text{ for all } w \in H_D^1(\Omega).\label{eq:pm:weakone}
\end{equation}
This weak form corresponds to the Firedrake/UFL code
\begin{Verbatim}[fontsize=\small]
    F = ( k * rho * dot(grad(c * rho + rho * g * z), grad(w)) ) * dx
\end{Verbatim}

\item This is a small modification of case 1.  We keep the top, bottom, and right sides the same, but suppose that there is nonzero flux into the left side:
\begin{subequations}
\begin{align}
\rho &= 1 & &(4=\text{top}) \label{eq:twobc:dirichlet} \\
\bq \cdot \bn = - \frac{k}{\mu} \grad(c\rho + \rho g z) \cdot \bn &= 0 & &(2,3=\text{right},\text{bottom}) \label{eq:twobc:noflux} \\
\bq \cdot \bn = - \frac{k}{\mu} \grad(c\rho + \rho g z) \cdot \bn &= q_1(z) & &(1=\text{left}) \label{eq:twobc:givenflux}
\end{align}
\end{subequations}
Now deriving the weak form from \eqref{eq:pm:preliminaryweak} we pick up a boundary integral:
\begin{equation}
\int_\Omega k \rho \grad\left(c\rho + \rho g z\right) \cdot \grad w + \int_{\partial_1\Omega} \mu \rho q_1 w = 0 \qquad \text{ for all } w \in H_D^1(\Omega).\label{eq:pm:weaktwo}
\end{equation}
Note that $q_1(z) = \bq \cdot \bn < 0$ corresponds to injecting gas into the left side, because $\bn = -\bx$ along that side.

For mildly interesting appearance we might set
	$$q_1(z) = \begin{cases} -c_1, & 0.2 < z < 0.4 \\ 0, & \text{otherwise} \end{cases}$$
for some value $c_1>0$.
\end{enumerate}

In the above cases the various constants are set to one, in the absence of preferred values: $k=1$, $c=1$, $\mu=1$, $g=1$.



\section{Numerical solution of the finite element equations}

Basically it is Newton's method, which is managed by the SNES component of PETSc.  Some useful options are to use the default ``line search'' Newton solver \texttt{newtonls}, and to ask for some feedback on the success/failure of the Newton steps, but to continue to solve the linear Newton step equations by LU decomposition:
\begin{Verbatim}[fontsize=\small]
    solve(F == 0, rho, bcs=[BCs],
          solver_parameters = {'snes_type': 'newtonls',
                               'snes_monitor': None,
                               'snes_converged_reason': None,
                               'ksp_type': 'preonly',
                               'pc_type': 'lu'})
\end{Verbatim}
Much more could be said.

\end{document}

